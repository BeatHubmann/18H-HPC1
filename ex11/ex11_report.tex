\documentclass[11pt,a4paper]{article}

% These are extra packages that you might need for writing the equations:
\usepackage{amsmath}
\usepackage{amsfonts}
\usepackage{amssymb}
\usepackage{booktabs}
\usepackage{hyperref}
\usepackage{listings}
\usepackage{xcolor}
\usepackage{graphicx}
\usepackage{subfig}
\usepackage{float}
\usepackage{pdfpages}

\lstset {language=C++,
		 basicstyle=\ttfamily,
         keywordstyle=\color{blue}\ttfamily,
         stringstyle=\color{red}\ttfamily,
         commentstyle=\color{purple}\ttfamily,
         morecomment=[l][\color{magenta}]{\#},
       	 basicstyle=\tiny}

% You need the following package in order to include figures in your report:
\usepackage{graphicx}

% With this package you can set the size of the margins manually:
\usepackage[left=2cm,right=2cm,top=2cm,bottom=2cm]{geometry}

\begin{document}

% Enter the exercise number, your name and date here:
\noindent\parbox{\linewidth}{
 \parbox{.25\linewidth}{ \large HPCSE I, Exercise 11 }\hfill
 \parbox{.5\linewidth}{\begin{center} \large Beat Hubmann \end{center}}\hfill
 \parbox{.2\linewidth}{\begin{flushright} \large Dec 15, 2018 \end{flushright}}
}
\noindent\rule{\linewidth}{2pt}

\includepdf[scale=0.8, pages=1, pagecommand=\section{Question 1: Diffusion in 2D using ADI scheme}\subsection{a)}]{q1a.pdf}

Both steps end up as tridiagonal matrices, which makes them ideal for the proposed
Thomas algorithm (aka tridiagonal matrix algorithm, TDMA). This is even more efficient than
using spare matrices as all is needed are three onedimensional vectors which will even
fit in the cache for a lot of problem sizes. The Thomas algorithm as such is inherently serial by design,
but that isn't overly relevant considering its light weight in computation effort and memory requirement.

\subsection{b)}

Done as instructed and submitted. For illustrative purposes, the starting conditions are plotted in figure~
\ref{fig:0_1}.

\begin{figure}[ht]
    \begin{center}
    \includegraphics[scale=1.0]{density0000.png} 
    \end{center}
    \caption{$\rho$ on $\Omega= (-1,1) \times (-1, 1), L=2, N=256, \Delta t = 10^{-6}s, D=1$ with Dirichlet boundary conditions: $\partial\Omega = 0$}
    \label{fig:0_1}
    \end{figure}

\subsection{c)}

Done as instructed and submitted where doable. Doing the same using MPI would involve a lot of communication
as the sweep directions alternates within each time step and thus would involve sharing/distributing data
after each sweep. Considering the fact that solving the tridiagonal systems can be done efficiently in 
terms of computations and memory, the communication overhead for MPI would hardly seem appropriate.


\subsection{d)}

The requested plot is shown in figure~\ref{fig:1}.

\begin{figure}[ht]
\begin{center}
\includegraphics[scale=1.0]{figure1.eps} 
\end{center}
\caption{approximative $\int_{\Omega}\rho$ for $\Omega= (-0.5,0.5) \times (-0.5, 0.5), L=1, N=256, \Delta t = 10^{-6}s, D=1$ with Dirichlet boundary conditions: $\partial\Omega = 0$}
\label{fig:1}
\end{figure}


\end{document}
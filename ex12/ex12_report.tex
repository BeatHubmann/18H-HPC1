\documentclass[11pt,a4paper]{article}

% These are extra packages that you might need for writing the equations:
\usepackage{amsmath}
\usepackage{amsfonts}
\usepackage{amssymb}
\usepackage{booktabs}
\usepackage{hyperref}
\usepackage{listings}
\usepackage{xcolor}
\usepackage{graphicx}
\usepackage{subfig}
\usepackage{float}
\usepackage{pdfpages}

\lstset {language=C++,
		 basicstyle=\ttfamily,
         keywordstyle=\color{blue}\ttfamily,
         stringstyle=\color{red}\ttfamily,
         commentstyle=\color{purple}\ttfamily,
         morecomment=[l][\color{magenta}]{\#},
       	 basicstyle=\tiny}

% You need the following package in order to include figures in your report:
\usepackage{graphicx}

% With this package you can set the size of the margins manually:
\usepackage[left=2cm,right=2cm,top=2cm,bottom=2cm]{geometry}

\begin{document}

% Enter the exercise number, your name and date here:
\noindent\parbox{\linewidth}{
 \parbox{.25\linewidth}{ \large HPCSE I, Exercise 12 }\hfill
 \parbox{.5\linewidth}{\begin{center} \large Beat Hubmann \end{center}}\hfill
 \parbox{.2\linewidth}{\begin{flushright} \large Dec 24, 2018 \end{flushright}}
}
\noindent\rule{\linewidth}{2pt}

\section{Question 1: 2D Diffusion using Particle Strenght Exchange (PSE)}

\subsection{a)}
Done as instructed.

\subsection{b)}
Done as instructed and submitted.

\subsection{c)}
Done as instructed and submitted.

\subsection{d)}

Given $N$ particles distributed random-uniformly on a 2D domain $[0,1)^2$, the distance between
particles in each direction can be estimated as $h \simeq \frac{1}{\sqrt{N}}$.\\

Running the experiment with kernel spreads $\varepsilon$ much larger or much smaller than $h$ leads to negligible exchange
between particles over the same given timeframe:
\begin{itemize}
    \item If $\varepsilon \ll h$, any neighbours are likely outside a kernel's width.
    \item If $\varepsilon \gg h$, a large amount of neighbours fall within the width of a kernel.
\end{itemize}

On the other hand, running the experiment with $\varepsilon \simeq h$ yields the expected 'text book' result.


% \begin{figure}[ht]
% \begin{center}
% \includegraphics[scale=1.0]{figure1.eps} 
% \end{center}
% \caption{approximative $\int_{\Omega}\rho$ for $\Omega= (-0.5,0.5) \times (-0.5, 0.5), L=1, N=256, \Delta t = 10^{-6}s, D=1$ with Dirichlet boundary conditions: $\partial\Omega = 0$}
% \label{fig:1}
% \end{figure}


\end{document}